\newpage
\section{Partículas}
\noindent 
Para empezar este capítulo, es importante tener claro al menos los conceptos básicos de las partículas y sus comportamientos, pues, es con estos conceptos son con los cuales estaremos trabajando en la computación cuántica, este es un resumen de todo lo básico y necesario, se puede estudiar a más profundidad si así lo desean, recomiendo en tal caso leerse el libro “Atomic Theory and the Description of Nature” de Niels Bohr
\subsection*{¿Qué es un electrón?}
\noindent 
El electrón es una partícula subatómica con carga eléctrica negativa de magnitud $-e$, donde $e = 1.602 \times 10^{-19} \, \text{C}$. Es una de las partículas fundamentales según el modelo estándar de la física de partículas y no tiene estructura interna conocida, es decir, no está compuesta por partículas más pequeñas. Posee una masa de:

\[
m_e = 9.109 \times 10^{-31} \, \text{kg}.
\]
\noindent 
El electrón participa en interacciones electromagnéticas y es responsable de la mayoría de las propiedades químicas de los átomos, ya que forma los enlaces químicos y define la configuración electrónica, existen muchas más partículas subatómicas, las cuales no mencionaremos en esta investigación pues nos estaríamos alejando del tema principal.

\subsection*{¿Qué es un fotón?}
\noindent 
El fotón es la partícula elemental de la luz y de todas las formas de radiación electromagnética. Es un bosón sin masa que actúa como el cuanto de energía del campo electromagnético. Su energía está relacionada con la frecuencia de la radiación según:

\[
E = h v,
\]
\noindent 
$v$ es la frecuencia de la onda electromagnética. Además, el fotón tiene momento lineal dado por:

\[
p = \frac{h}{\lambda},
\]
\noindent 
$\lambda$ es la longitud de onda. Los fotones no tienen carga eléctrica y viajan a la velocidad de la luz en el vacío, $c = 3.00 \times 10^8 \, \text{m/s}$.

\subsection*{¿Qué es un átomo?}
\noindent 
El átomo es la unidad básica de la materia que conserva las propiedades químicas de un elemento. Está compuesto por un núcleo central y una nube de electrones. El núcleo, a su vez, contiene protones y neutrones:

\begin{itemize}
    \item \textbf{Protones:} Partículas (partícula subatómica) con carga positiva igual a $+e$ y masa $m_p = 1.673 \times 10^{-27} \, \text{kg}$.
    \item \textbf{Neutrones:} Partículas (partícula subatómica) neutras (sin carga) con una masa ligeramente mayor que la del protón, $m_n = 1.675 \times 10^{-27} \, \text{kg}$.
\end{itemize}
\noindent 
Los electrones orbitan alrededor del núcleo en niveles de energía cuantizados, definidos por los principios de la mecánica cuántica. Los átomos de plata tienen 47 electrones: dos en la órbita más interna, ocho en la siguiente, luego dieciocho en cada uno de los dos niveles siguientes. Esto deja un único electrón en la órbita más externa. Los átomos de oro tiene 79 electrones, a diferencia del átomo de plata, este tiene 6 niveles, en el cual, el primer nivel cuenta con dos electrones, ocho en el segundo nivel, dieciocho  en el tercero, 32 en el cuarto, dieciocho  en el quinto e igual que el átomo de plata, el último nivel o la órbita más externa cuenta con un único electrón.

\subsection*{¿Qué es una partícula?}
\noindent 
En física, una partícula es una entidad que puede describirse mediante propiedades como masa, carga, y momento. Las partículas subatómicas se dividen en:

\begin{itemize}
    \item \textbf{Fermiones:} Constituyen la materia (electrones, protones, neutrones, quarks, etc.) y obedecen el principio de exclusión de Pauli.
    \item \textbf{Bosones:} Median las interacciones fundamentales (fotones, gluones, bosones $W$ y $Z$, etc.).
\end{itemize}

\subsection*{¿De qué se compone un núcleo?}
\noindent 
El núcleo del átomo está compuesto por protones y neutrones, colectivamente llamados nucleones. La fuerza nuclear fuerte, mediada por los gluones, mantiene unidos a los nucleones, contrarrestando la repulsión eléctrica entre los protones. El número de protones define el elemento químico (número atómico $Z$), mientras que el número de neutrones determina los isótopos de dicho elemento.


\subsection*{Modelo atómico de Niels Bohr}
\noindent 
hablaremos un poco sobre el modelo atómico propuesto en 1913 por el físico danés Niels Bohr, el cual representó una importante evolución en la descripción del átomo al introducir formalmente el concepto de cuantización de energía a través de postulados fundamentales. Este modelo se basó en los principios de la mecánica cuántica y la teoría del espectro atómico, estableciendo que los electrones pueden ocupar únicamente ciertas órbitas alrededor del núcleo, denominadas niveles de energía cuantizados, en las cuales no emiten radiación electromagnética debido a su movimiento acelerado. Este enfoque resolvía problemas inherentes al modelo clásico de Rutherford, que no podía explicar la estabilidad de los átomos. Además, Bohr formuló que los átomos emiten o absorben energía en forma de fotones cuando los electrones realizan transiciones entre estos niveles de energía, lo que permite interpretar con precisión los espectros de emisión y absorción característicos de cada elemento, según la relación:

\[
E = hv
\]
\noindent 
$h$ es la constante de Planck y $v$ la frecuencia del fotón emitido o absorbido.\\[0.5em]
\noindent 
En este modelo, los electrones giran alrededor en órbitas del núcleo del átomo sin irradiar energía. Ahora bien, no toda órbita para el electrón está permitida: ellos ocupan la órbita de menor energía posible o la órbita más cercana posible al núcleo. El electrón solo emite o absorbe energía en los saltos de una órbita permitida a otra. Al realizar ese cambio, emite o absorbe un fotón cuya energía es la diferencia de energía entre ambos niveles, dada por:

\[
\Delta E = E_f - E_i = h v
\]
\noindent 
$E_f$ y $E_i$ son las energías de los niveles final e inicial, respectivamente.\\[0.5em]
\noindent 
Según Bohr, las órbitas permitidas están determinadas por la cuantización del momento angular del electrón, descrita por la fórmula:

\[
L = n \hbar = n \frac{h}{2\pi}, \quad n = 1, 2, 3, \ldots
\]
\noindent 
$L$ es el momento angular, $n$ es el número cuántico principal y $\hbar = \frac{h}{2\pi}$ es la constante de Planck reducida.\\[0.5em]
\noindent 
La energía total de un electrón en una órbita permitida está dada por:

\[
E_n = -\frac{Z^2 e^4}{8 \varepsilon_0^2 h^2 n^2}
\]
\noindent 
Este modelo permitió explicar el espectro de emisión del hidrógeno, cuya longitud de onda está dada por la fórmula de Rydberg:

\[
\frac{1}{\lambda} = R_H \left( \frac{1}{n_i^2} - \frac{1}{n_f^2} \right)
\]
\subsection*{Interacciones Fundamentales}
\noindent
En el universo, las partículas subatómicas interactúan a través de cuatro fuerzas fundamentales que determinan la estructura y dinámica de la materia. Estas fuerzas son:\\[0.5em]
\noindent
\textbf{Fuerza Gravitacional:} Actúa entre masas y es responsable de fenómenos como la formación de planetas y galaxias. Aunque universal, su efecto a escala subatómica es despreciable comparado con las demás fuerzas.\\[0.5em]
\noindent
\textbf{Fuerza Electromagnética:} Responsable de la interacción entre partículas con carga eléctrica. Es mediada por fotones y describe fenómenos como la formación de enlaces químicos y las propiedades ópticas de los materiales.\\[0.5em]
\noindent
\textbf{Fuerza Nuclear Fuerte:} Une los quarks dentro de los protones y neutrones y mantiene el núcleo atómico unido, superando la repulsión electrostática entre protones. Es mediada por los gluones.\\[0.5em]
\noindent
\textbf{Fuerza Nuclear Débil:} Responsable de procesos de desintegración nuclear, como la emisión beta. Es mediada por los bosones $W^\pm$ y $Z^0$ y juega un papel clave en la física de partículas y la cosmología.

\subsection*{Partículas según el Modelo Estándar}
\noindent
El modelo estándar de la física de partículas clasifica las partículas fundamentales en dos categorías principales:\\[0.5em]
\noindent
\textbf{Fermiones:} Constituyen la materia. Obedecen al principio de exclusión de Pauli y tienen espín semi-entero $(\frac{1}{2})$.
    \begin{itemize}
        \item \textbf{Quarks:} Se combinan para formar hadrones como protones y neutrones. Hay seis tipos (sabores): arriba ($u$), abajo ($d$), encanto ($c$), extraño ($s$), cima ($t$) y fondo ($b$).
        \item \textbf{Leptones:} Incluyen el electrón ($e^-$), el muón ($\mu^-$), el tauón ($\tau^-$), y sus correspondientes neutrinos ($\nu_e$, $\nu_\mu$, $\nu_\tau$).
    \end{itemize}
\noindent
\textbf{Bosones:} Median las interacciones fundamentales. Tienen espín entero.
    \begin{itemize}
        \item \textbf{Fotón ($\gamma$):} Mediador de la fuerza electromagnética.
        \item \textbf{Gluones ($g$):} Mediadores de la fuerza fuerte.
        \item \textbf{Bosones $W^\pm$ y $Z^0$:} Mediadores de la fuerza débil.
        \item \textbf{Bosón de Higgs ($H$):} Responsable de dar masa a las partículas a través del mecanismo de Higgs.
    \end{itemize}

\subsection*{Dualidad Onda-Partícula}
\noindent
Un concepto fundamental en la mecánica cuántica es la dualidad onda-partícula, que establece que las partículas pueden exhibir comportamientos tanto de partículas como de ondas. Esto se describe mediante:\\[0.5em]
\noindent
\textbf{Longitud de onda de De Broglie:}
    \[
    \lambda = \frac{h}{p},
    \]
    donde $h$ es la constante de Planck y $p$ es el momento lineal de la partícula.\\[0.5em]
\noindent
\textbf{Principio de incertidumbre de Heisenberg:}
    \[
    \Delta x \Delta p \geq \frac{\hbar}{2},
    \]
    que establece límites fundamentales en la precisión con la que se pueden medir simultáneamente la posición ($x$) y el momento ($p$) de una partícula.


